\section*{Introduction}

For the course ``Machine Learning in Practice'' we participated in the annual KDD-cup.
This year's KDD-Cup is about author-paper identification.
In this paper we present and discuss our approach to try to win this competition.
The methodology is illustrated using the data provided by the KDD-Cup.
Our approach involves three different methods to find features in the given data set.
We've used a probabilistic author-topic model\cite{steyvers2004probabilistic} to extract topics from  the data, co-authorship graphs to get extra features from social connections between authors and we have used some simple features provided by the benchmark package provided by the competition organisers.
These features are then used together to train a random forest classifier.

The outline of the paper is as follows: in Section~\ref{sec:background} we describe the scientific background of our methods, then describe our implementation details in Sections~\ref{sec:graph-implementation},~\ref{sec:text-implementation},~\ref{sec:smaller-features-implementation} and \ref{sec:classifier-implementation}. Finally, we present our results in Section~\ref{sec:results}.
