%\subsection*{Text analyses \label{sec:imp-text}}

In section \ref{sec:text-background} we introduced a model to create topics from a large collection of text. We adjusted that approach to the problem we needed to solve. We did not have a lot of text from each of the papers. From each paper we only knew its title and keywords.
We did not choose to use the keywords.
The keywords contained a lot of bias and parsing them in a usable format proved to be difficult.
We instead used the titles of the journal or conference the paper was published in.
This method has two advantages:

\begin{itemize}
\item[1] We have more terms to identify a paper.
\item[2] More terms are related to each other. If for example the "Automatic detection of acute bacterial pneumonia from chest X-ray reports" was published in a journal containing the terms ``machine learning'', then the terms ``machine learning'' and ``X-ray'' are considered to be related. 
\end{itemize}

We now have a sentence $s$ describing the content of paper $p$.
Next, all the diacritics, stopwords (in multiple languages), interpunction and double terms are removed from sentence $s$.
This saves a lot of entries in the dictionary\cite{chowdhury2010introduction}.
After processing sentence $s$, all terms that co-occur in the same sentence are counted and the result is stored in matrix $M$.
A simple example with some of the terms in sentence $s$ is worked out in tables \ref{tab:before} and \ref{tab:after}.

\begin{table}
	\begin{center}
	

\begin{tabular}{|l|l|l|l|l|}
\hline
	 	& acute  & learning & machine &  x-ray \\ \hline
acute 	&	0 	& 	0 &	 0 &  1	 \\ \hline
learning&	0	&	0 &	 0 &  0	 \\ \hline
machine &	0	&	0 &	 0 &  0	 \\ \hline
x-ray	&	1	&	0 &	 0 &  0	 \\ \hline

\end{tabular} 
	\end{center}
	\caption{A paper titled ``Automatic detection of acute bacterial pneumonia from chest X-ray reports'', split up into a word matrix where related / connected words are counted.}
	\label{tab:before}
\end{table}


\begin{table}
	\begin{center}

\begin{tabular}{|l|l|l|l|l|}
\hline
	 	& acute  & learning & machine &  x-ray \\ \hline
acute 	&	0 	& 	2 &	 3 &  1	 \\ \hline
learning&	2	&	0 &	 4 &  1	 \\ \hline
machine &	3	&	4 &	 0 &  1	 \\ \hline
x-ray	&	1	&	1 &	 1 &  0	 \\ \hline
\end{tabular} 

	\end{center}
\caption{The same paper, added to a matrix which already had some word counts for `machine' and `learning', which we decided to link to based on the journal/conference where the paper was contributed.}
	\label{tab:after}
\end{table}

Note that matrix $M$ is symmetric and very sparse. We used this to our advantage and stored this matrix efficiently.


\subsubsection*{Clustering}

In the previous subsection we created matrix $M$. Each row is a separate vector $v$ containing information form which terms co-occurs with a another term. Each vector $v$ is then clustered into one of the $c$ clusters with k-means. In our case, we choose  $c = 20$. The number 20 is a wild guess. In the original paper \cite{steyvers2004probabilistic}, the authors use 300 clusters, but our computers were not able to process that number of clusters. Each cluster now contains terms that co-occur a lot. We call a cluster a topic, just as the original paper. As a distance measure we choose \emph{cosine similarity}. Cosine similarity measures the distance between two vectors. The cosine similarity has a big advantage over the Euclidean distance \cite{chowdhury2010introduction}.  

Not that the authors use EM to cluster matrix $M$. This gives the authors a probability of each term for each cluster. because we used k-means we do not have that information anymore. We only which term is in a certain cluster. In our case, each term can only be in a single cluster.

\subsubsection*{Scoring}

At this point we have 20 clusters. Each cluster $c$ contains several terms that that form a single topic. We then score each paper-author combination against the topics. For each  paper-author combination we created sentence $s$ just as in section \ref{sec:imp-text}. For each term in sentence $s$ we counted the number of occurrences in a topic. This is then used as latent variable for classifying paper-
author couples.

\begin{multicols}{2}
\[ \delta_c(t) =\left\{ \begin{matrix} 1 & \mbox{if t is present in c} \\ 0 & \mbox{otherwise} \end{matrix} \right.\]

\[ score_c = \sum_{t \in s} \delta_c(t) \]
\end{multicols}

\subsubsection*{Optimizing with Tf-Idf}

Thus far, scoring has hinged on how much a term occurs. Common words have a higher score and contribute more towards the final classification. But common words do not say a lot about each sentence\cite{chowdhury2010introduction}. To compensate the higher score for common terms we use a information retreival technique called TF-IDF. Instead of just counting the number of occurrences of a term in a cluster, we assign a TF-IDF for each term and add those together. We simplified TF-IDF for our own use. because a term can only be present once each sentence, we only use IDF. The score for each cluster is now computed as:


\begin{multicols}{2}
\[ \gamma_c(t) =\left\{ \begin{matrix} \mbox{Tf-Idf}(t) & \mbox{if t is present in c} \\ 0 & \mbox{otherwise} \end{matrix} \right.\]


\[ score_c = \sum_{t \in s} \gamma_c(t) \]
\end{multicols}
